\documentclass[11pt]{article}
\usepackage[utf8]{inputenc}
\usepackage{amsmath,amssymb,amsthm}
\usepackage{hyperref}
\usepackage[capitalise,noabbrev]{cleveref}
\usepackage{geometry}
\usepackage{booktabs}
\usepackage{graphicx}
\geometry{margin=1in}

% Theorem environments
\newtheorem{theorem}{Theorem}[section]
\newtheorem{proposition}[theorem]{Proposition}
\newtheorem{corollary}[theorem]{Corollary}
\newtheorem{lemma}[theorem]{Lemma}
\theoremstyle{definition}
\newtheorem{definition}[theorem]{Definition}
\newtheorem{conjecture}[theorem]{Conjecture}
\theoremstyle{remark}
\newtheorem{remark}[theorem]{Remark}

\title{Where the Lovelock Bridge Breaks:\\
Negative Results and New Directions for Connecting\\
Discrete and Continuous Spacetime Emergence}

\author{Max Zhuravlev\thanks{Independent researcher. Email: \texttt{max@vibecodium.ai}}}

\date{February 2026}

\begin{document}

\maketitle

\begin{abstract}
We examine the hypothesis that Wolfram's hypergraph physics and Vanchurin's
neural network cosmology can be connected via Lovelock's uniqueness theorem.
The proposed bridge runs: causal invariance $\to$ discrete covariance (Gorard 2020)
$\to$ continuum limit $\to$ diffeomorphism invariance $\to$ Lovelock uniqueness $\to$
Einstein equations, potentially constraining Vanchurin's Onsager tensor symmetries.
We report three negative results that obstruct this bridge:
(1)~numerical evidence that the continuum limit fails for all dynamically
nontrivial hypergraph rewrite rules tested (500~rules, up to
$N = 18{,}508$ nodes; no connected bounded-degree rule converges to
positive Ollivier--Ricci curvature);
(2)~a fundamental barrier between discrete permutation-type symmetries and the
continuous Lorentz group required by Lovelock's theorem;
(3)~the observation that Vanchurin's current Type~II framework (2025)
bypasses the continuum limit entirely by working in continuous trainable
parameter space.
Having documented why the direct bridge fails, we turn to constructive results
within Vanchurin's Type~II metric framework.
We prove an exact formula for the critical inverse temperature~$\beta_c$ at
which the observer metric transitions between Lorentzian and Riemannian
signatures, establish that the mass tensor equals the square of the Fisher
matrix for exponential family models, and identify a spectral purity condition
under which mass and Fisher tensors become proportional.
We compare our signed-edge construction for Lorentzian signature with
Vanchurin's non-principal square root mechanism and prove they are
non-equivalent.
Additionally, we prove the \emph{Tree Fisher Identity}: the Ising Fisher
matrix on tree graphs equals $\operatorname{sech}^2(J)$ times the identity,
and derive a rigorous bound on off-diagonal Fisher entries that decay as
$O(\tanh^{g-2}(J))$ with graph girth~$g$, with an exact adjacent-edge
covariance formula on cycles.
We prove that any positive definite diagonal matrix (the structural
case for tree observers) selects Lorentzian signature ($q = 1$) with
explicit margin, and that this dominance is stable under perturbation
(robust to off-diagonal ratio ${\sim}0.9$); however, the signed-edge
construction H1$'$ creates Lorentzian \emph{accessibility} without
preferential selection (spectral gap degeneracy).
Universality tests across $q$-state Potts models ($q = 2$--$5$,
72~configurations) confirm that these spectral properties extend beyond
the Ising case to all discrete spin models.
These results contribute to the formal development of Type~II metric theory
for observers in learning systems.

\medskip
\noindent\textbf{Keywords}: Wolfram physics, neural network cosmology,
Lovelock theorem, negative results, Lorentzian signature, Fisher information,
Type~II metrics

\medskip
\noindent\textbf{arXiv category}: gr-qc (cross-list: hep-th)
\end{abstract}

%----------------------------------------------------------------------
\section{Introduction}
\label{sec:introduction}
%----------------------------------------------------------------------

Two recent cosmological research programs have independently
derived gravitational physics from novel axioms, yet have never been
formally compared at a technical level.

\textbf{Wolfram Hypergraph Physics}~\cite{Wolfram2020,Gorard2020}.
Spacetime and gravity emerge from hypergraph rewriting with causal
invariance (CI) as the fundamental axiom.
Gorard~\cite{Gorard2020} proved that CI is equivalent to a discrete
form of general covariance for hypergraph evolution.
Under a conjectured continuum limit, this discrete covariance would
yield smooth diffeomorphism invariance.

\textbf{Vanchurin Neural Network Cosmology}~\cite{Vanchurin2020,Vanchurin2025,Vanchurin2025CGD,Vanchurin2024EFT}.
Gravity emerges from entropy production and learning dynamics in trainable
neural networks.
In Vanchurin's original (Type~I) framework~\cite{Vanchurin2020}, the metric
lives in the space of non-trainable variables, requiring a discrete-to-continuous
transition analogous to the Wolfram program's continuum limit.
In the current Type~II framework~\cite{Vanchurin2025,Vanchurin2025CGD},
the metric lives in the continuous space of trainable parameters, bypassing
this barrier entirely.

\subsection{The Lovelock bridge hypothesis}
\label{sec:bridge-hypothesis}

Lovelock's uniqueness theorem~\cite{Lovelock1971} establishes that in four
spacetime dimensions, the only symmetric, divergence-free, rank-2 tensor
constructed from the metric and its first two derivatives is
$a\,G^{\mu\nu} + b\,g^{\mu\nu}$, where $G^{\mu\nu}$ is the Einstein
tensor.
This theorem suggested a natural bridge between the two programs: if
causal invariance implies diffeomorphism invariance (via a continuum
limit), and if Vanchurin's gravitational dynamics satisfy Lovelock's
hypotheses, then both programs are forced to produce Einstein's equations
as the unique gravitational dynamics.

In his original paper, Vanchurin~\cite{Vanchurin2020} noted that the
symmetries of his Onsager tensor (Eq.~93 in \cite{Vanchurin2020}) are
introduced phenomenologically, and asked whether they could be
``derived from first principles.''
The Lovelock bridge was conceived as a potential answer.

\subsection{Summary of results}
\label{sec:summary}

The bridge hypothesis fails.
This paper documents why, and presents constructive results that emerged
from the investigation.

\textbf{Negative results} (\cref{sec:failure}):
\begin{enumerate}
\item The continuum limit is falsified numerically for all 500~rules
  tested (\cref{sec:numerical}).
\item A fundamental discrete-to-continuous symmetry barrier prevents
  passage from Gorard's discrete covariance to smooth diffeomorphism
  invariance (\cref{sec:symmetry-barrier}).
\item Vanchurin's Type~II framework bypasses the problem that the bridge
  was designed to address (\cref{sec:lovelock-learning}).
\end{enumerate}

\textbf{Constructive results} (\cref{sec:typeII,sec:mass-fisher,sec:lorentzian,sec:spectral-gap-selection}):
\begin{enumerate}
\item The Type~II metric decomposes as $g_{\mu\nu} = M_{\mu\nu} +
  \beta\,F_{\mu\nu}$, where $M$ is a structural inertia tensor and $F$ the
  Fisher information metric (\cref{sec:metric-decomposition}).
\item For exponential family models, $M = F^2$ exactly
  (\cref{sec:M-equals-F-squared}).
\item Under a Structural Reflection Condition (SRC), perfect Good Regulators
  satisfy $M = \kappa\,F$, making the metric conformally Fisher
  (\cref{sec:spectral-purity}).
\item A critical inverse temperature $\beta_c$ governs the
  Lorentzian--Riemannian transition, with exact formula
  $\beta_c = -d_1$ where $d_1$ is the most negative eigenvalue of the
  Fisher-weighted mass tensor (\cref{sec:beta-c-theorem}).
\item Our signed-edge construction (H1$'$) and Vanchurin's non-principal
  square root are non-equivalent (\cref{sec:comparison-vanchurin}).
\item The Tree Fisher Identity shows that the Ising Fisher matrix on
  tree graphs equals $\operatorname{sech}^2(J) \cdot I_m$; the
  Near-Diagonal Fisher Theorem extends this to sparse graphs with
  off-diagonal decay $O(\tanh^g(J))$; and a spectral gap selection
  mechanism demonstrates that sparse observer graphs overwhelmingly
  prefer Lorentzian ($q = 1$) signature, verified for dimensions up
  to $n = 20$ (\cref{sec:spectral-gap-selection}).
\end{enumerate}

\subsection{Relation to prior work}
\label{sec:prior-work}

Gorard~\cite{Gorard2020} established the rigorous discrete covariance
result; we do not question this theorem but show that it does not extend
to smooth diffeomorphism invariance.
Lovelock~\cite{Lovelock1971} proved the uniqueness theorem that our
bridge invokes; the theorem is correct but its hypotheses are not met by
the Wolfram program in its current state.
Matsueda~\cite{Matsueda2013} derived Einstein equations from the Fisher
information metric, providing an independent information-theoretic path
to gravity.
For information-geometric criticality context in statistical systems we use
the thermodynamic-geometry program of Ruppeiner and follow-up finite-size
studies~\cite{Ruppeiner1979,Ruppeiner1995,Janke2004,BrodyRitz2003}.
Amari~\cite{Amari1998} established the natural gradient as the unique
covariant gradient descent; this is standard and not claimed as novel.
Rao~\cite{Rao1945} introduced the Fisher metric as a Riemannian metric
on statistical manifolds.
Ollivier~\cite{Ollivier2009} introduced the Ricci curvature for metric
spaces used in our numerical tests.
Conant and Ashby~\cite{ConantAshby1970} proved the Good Regulator
theorem; Virgo et al.~\cite{Virgo2025} reformulated it for embodied
agents in a form we use in \cref{sec:spectral-purity}.
Chiribella et al.~\cite{Chiribella2011} derived quantum theory from
operational axioms; this framework is used in our companion
paper~\cite{companion_qm} (in preparation).

%----------------------------------------------------------------------
\section{The Lovelock Bridge: Statement and Failure}
\label{sec:failure}
%----------------------------------------------------------------------

\subsection{Precise logical chain}
\label{sec:logical-chain}

The bridge argument consists of seven steps, labeled L1--L7.
We state each, indicate its status, and assign a conditional probability
$P(\text{L}_{k} \mid \text{L}_{k-1})$.

\begin{enumerate}
\item[\textbf{L1.}] \textit{Causal invariance holds for the hypergraph
  rewriting system.}
  Status: imposed by hypothesis.
  $P(\text{L1}) = 0.95$ (CI is not generic; it holds for only a minority
  of rules~\cite{Wolfram2020,Gorard2020}).

\item[\textbf{L2.}] \textit{CI $\Rightarrow$ discrete general covariance.}
  Status: \textbf{proven} (Gorard 2020~\cite{Gorard2020}).
  $P(\text{L2} \mid \text{L1}) = 0.95$.

\item[\textbf{L3.}] \textit{Discrete covariance $\Rightarrow$
  diffeomorphism invariance in a continuum limit.}
  Status: \textbf{falsified numerically} (500~rules, see \cref{sec:numerical}).
  $P(\text{L3} \mid \text{L2}) \approx 0.05$.

\item[\textbf{L4.}] \textit{The continuum equations satisfy Lovelock's
  hypotheses: locality, at most second-order derivatives, metric-based.}
  Status: \textbf{not established} for learning dynamics
  (see \cref{sec:lovelock-learning}).
  $P(\text{L4} \mid \text{L3}) \approx 0.30$.

\item[\textbf{L5.}] \textit{The emergent energy-momentum tensor is
  divergence-free.}
  Status: not automatic for general learning dynamics.
  $P(\text{L5} \mid \text{L4}) \approx 0.40$.

\item[\textbf{L6.}] \textit{$D = 4$ spacetime dimensions.}
  Status: assumed (observational, not derived from CI).
  $P(\text{L6} \mid \text{L5}) = 0.99$.

\item[\textbf{L7.}] \textit{Lovelock uniqueness $\Rightarrow$ Einstein
  equations $\Rightarrow$ constrains Onsager tensor.}
  Status: theorem-level, conditional on L1--L6.
  $P(\text{L7} \mid \text{L6}) = 0.99$.
\end{enumerate}

\begin{remark}[End-to-end probability]
\label{rem:probability}
The end-to-end probability of the bridge holding is
\begin{equation}
\label{eq:bridge-probability}
P(\text{Einstein} \mid \text{CI}) \;=\;
\prod_{k=1}^{7} P(\text{L}_k \mid \text{L}_{k-1})
\;\approx\; 0.95 \times 0.95 \times 0.10 \times 0.30 \times 0.40 \times 0.99 \times 0.99
\;\approx\; 0.011\,.
\end{equation}
That is, conditional on causal invariance, the probability that the full
Lovelock bridge delivers Einstein equations is approximately~$1\%$.
The bottleneck is Step~L3 (continuum limit), followed by L4 and~L5.
Steps~L4 and~L5 are subjective estimates reflecting the author's
assessment of how likely learning dynamics satisfy Lovelock's
hypotheses; they are not derived from numerical experiments or
literature consensus.
\end{remark}

\subsection{Numerical investigation of the continuum limit}
\label{sec:numerical}

We tested Wolfram rewrite rules for convergence of Ollivier--Ricci
curvature~\cite{Ollivier2009} in the large-$N$ limit.
An initial survey of 13~rules was extended to a systematic search of
500~rules (282~binary, 218~ternary) with evolution up to $N = 18{,}508$
nodes. For each rule, we computed the mean Ollivier--Ricci
curvature $\bar{\kappa}(N)$ of the spatial hypergraph at increasing sizes
and fitted $|\bar{\kappa}(N)| \sim N^\alpha$.

The results are summarized in \cref{tab:curvature} and visualized in \cref{fig:ollivier-ricci}.

\begin{table}[ht]
\centering
\caption{Ollivier--Ricci curvature convergence for 500~hypergraph rewrite
rules (282~binary, 218~ternary). ``Converge'' means
$\bar{\kappa}(N) \to \kappa_\infty \neq 0$ as $N \to \infty$.
The four bounded-degree rules show nonzero stable curvature but converge
to negative values ($-0.3$ to $-0.45$) with effective
dimension~$\sim 1.0$---graph sparsity artifacts, not manifold-like
geometry.}
\label{tab:curvature}
\begin{tabular}{@{}lccc@{}}
\toprule
Rule class & Rules tested & Converge to $\kappa \neq 0$ & Scaling \\
\midrule
Expanding (connected, $\Delta$-bounded) & 285 & 4/285${}^*$
    & $\kappa \to -0.3$ to $-0.45$ \\
Expanding (disconnected / hub) & incl.\ above & many
    & Tiling or star artifacts \\
Contracting / static   & 207 & trivially & Fixed at $N \leq 8$ \\
\midrule
\textbf{Total}         & \textbf{500} & \textbf{0/500}${}^{**}$ & \\
\bottomrule
\end{tabular}

\medskip\noindent
{\footnotesize ${}^*$Four bounded-degree rules stabilize at negative
curvature: $\kappa \approx -0.19, -0.35, -0.43, -0.43$ at
$N > 8{,}000$. These values lie between the tree value ($-0.5$)
and the cycle value ($0$), reflecting graph sparsity.
${}^{**}$No rule produces positive stable curvature in a connected,
bounded-degree graph.}
\end{table}

The pattern is clear: every rule with nontrivial expanding dynamics shows
$\bar{\kappa}(N) \sim 1/N$ or converges to a \emph{negative} curvature
reflecting graph sparsity rather than manifold-like geometry.
Rules with apparently ``stable positive curvature'' fall into two
categories: (i)~disconnected graphs tiled with identical small components
(sizes~$3$--$5$), where $\kappa$ measures the curvature of the repeated
motif, not a large-scale geometric property; (ii)~star/hub graphs with
a single vertex of degree $\sim N$, which are not manifold-like.

This is a stronger statement than ``the continuum limit is unproven.''
The continuum limit is \emph{empirically disconfirmed} for every
dynamically interesting rule in our sample.
We assign confidence ${>}\,95\%$ to this conclusion (500-rule survey,
three curvature measures, structural classification of all apparent
counterexamples).

The structural mechanism for the failure is: as a connected hypergraph
grows, new vertices are added at the boundary, diluting local clustering.
The triangle fraction decreases, driving Ollivier--Ricci curvature toward
the tree value of $-\alpha = -0.5$.
Only disconnected tiling avoids this dilution, but such configurations do
not constitute a continuum limit.

\begin{figure}[htbp]
\centering
\includegraphics[width=0.8\textwidth]{Fig_OllivierRicci_Curvature.png}
\caption{Ollivier--Ricci curvature convergence for hypergraph rewrite
rules. All dynamically nontrivial expanding rules show $\kappa \sim 1/N$
decay to zero (or convergence to negative values between the tree and
cycle limits) in the large-$N$ limit. Only static fixed-point rules
($N \leq 8$, shown in plateau) converge to nonzero positive curvature.
Extended survey: 500~rules tested (282~binary, 218~ternary), up to
$N = 18{,}508$ nodes.}
\label{fig:ollivier-ricci}
\end{figure}

\subsection{The discrete-to-continuous symmetry barrier}
\label{sec:symmetry-barrier}

Even if the continuum limit problem were resolved, a second barrier
remains.
Gorard's~\cite{Gorard2020} discrete covariance theorem establishes
invariance under \emph{permutation-type} transformations of the
hypergraph update order---these are elements of a finite symmetric group.
Lovelock's theorem requires diffeomorphism invariance under the
\emph{continuous} Lie group $\mathrm{Diff}(M)$.
The Lorentz group $\mathrm{SO}(3,1)$ is a continuous Lie group; no finite
set of discrete transformations generates it.

Coarse-graining of hypergraph dynamics can produce approximate spatial
symmetries (rotations, translations) in the large-$N$ regime, as
demonstrated by Wolfram model simulations~\cite{Wolfram2020}.
However, Lorentz \emph{boosts} are qualitatively different: they mix
spatial and temporal directions and require a continuous parameter
(rapidity).
No known coarse-graining procedure generates boosts from discrete
permutation symmetries.

The gap between discrete and continuous covariance may be fundamental
rather than technical.
This is not a new observation---it reflects a longstanding challenge in
all discrete approaches to quantum gravity---but its specific implications
for the Wolfram--Vanchurin bridge have not been previously documented.

\subsection{Lovelock assumptions and learning dynamics}
\label{sec:lovelock-learning}

Lovelock's theorem~\cite{Lovelock1971} requires four conditions:
(a)~general covariance, (b)~divergence-free energy-momentum tensor,
(c)~field equations at most second-order in metric derivatives,
(d)~$D = 4$ spacetime dimensions.
For learning dynamics~\cite{Vanchurin2020,Vanchurin2025}, these conditions
are not automatically satisfied:

\begin{enumerate}
\item General covariance holds only in the continuum limit, which is not
  established (Step~L3).
\item The divergence-free condition on Vanchurin's Onsager tensor is an
  assumption, not a derivation, in the Type~I framework.
\item Learning dynamics can produce higher-order equations when the loss
  landscape is sufficiently complex.
\item $D = 4$ is observationally motivated but not derived from any axiom
  in either program.
\end{enumerate}

However, within Vanchurin's Type~II framework, these objections are
largely resolved; see \cref{sec:typeII}.

\subsection{End-to-end probability analysis}
\label{sec:probability-analysis}

\Cref{rem:probability} estimates the end-to-end success probability at
approximately~$1\%$.
This makes the Lovelock bridge, in its original form, an implausible
argument.
The dominant failure modes are the continuum limit (L3, $P \approx 0.10$)
and the Lovelock hypothesis compatibility with learning dynamics
(L4--L5, combined $P \approx 0.12$).

We emphasize that this is not merely a gap in current knowledge: Step~L3
is actively falsified by the numerical evidence of \cref{sec:numerical},
and Steps~L4--L5 face the structural objections detailed
in~\cref{sec:lovelock-learning}.

%----------------------------------------------------------------------
\section{Vanchurin's Type~II Framework: Why It Bypasses the Problem}
\label{sec:typeII}
%----------------------------------------------------------------------

\subsection{Type~I versus Type~II}
\label{sec:type-I-vs-II}

Vanchurin's framework has undergone a fundamental evolution.

In the \textbf{Type~I} framework~\cite{Vanchurin2020}, the metric tensor
lives in the space of non-trainable variables~$X$ (analogous to the
spatial hypergraph in the Wolfram program).
This space is discrete, and recovering smooth geometry requires exactly
the continuum limit that our bridge depends on.

In the \textbf{Type~II} framework~\cite{Vanchurin2025,Vanchurin2025CGD},
the metric tensor lives in the space of trainable parameters~$Q$.
This space is continuous by construction---the parameters are
real-valued weights of the neural network---so no discrete-to-continuous
transition is needed.

The entire Lovelock bridge was designed to address a problem (the
passage from discrete to continuous covariance) that no longer exists in
the current version of one of the two programs being bridged.

\subsection{Covariant gradient descent}
\label{sec:CGD}

In the Type~II framework, the learning dynamics take the form of
covariant gradient descent (CGD)~\cite{Vanchurin2025CGD}:
\begin{equation}
\label{eq:CGD}
\dot{q}^\mu = -\gamma\, g^{\mu\nu}(q)\, F_\nu(q)\,,
\end{equation}
where $g^{\mu\nu}$ is the inverse metric on parameter space, $F_\nu =
-\langle \partial_\nu L \rangle$ is the averaged force from the loss
function~$L$, and $\gamma$ is a learning rate.
General covariance is built into this equation: it is manifestly tensorial
under reparametrizations of~$Q$.

\subsection{Direct derivation of Einstein equations}
\label{sec:einstein-derivation}

Vanchurin~\cite{Vanchurin2025} derives the Einstein equations through a
constructive procedure:
\begin{enumerate}
\item Consider an ensemble of $n$ learning agents, each with metric
  $g_{n,\mu\nu}$.
\item Define a local averaged metric via Gaussian-weighted ensemble
  average.
\item Construct the Einstein--Hilbert action from this ensemble metric.
\item Variation yields the vacuum Einstein equations:
  \begin{equation}
  \label{eq:einstein-vacuum}
  R_{\mu\nu} - \tfrac{1}{2}\, g_{\mu\nu}\, R = 0\,.
  \end{equation}
\end{enumerate}
This derivation does not invoke the Lovelock theorem.
It provides a constructive proof where our bridge provides only a
uniqueness argument.
Consequently, the Lovelock bridge, even if it could be completed, would
be \emph{confirmatory} rather than predictive: it would confirm a result
that Vanchurin already derives directly.

\subsection{Three regimes}
\label{sec:three-regimes}

Vanchurin's framework~\cite{Vanchurin2025} identifies three regimes via
$g \propto \kappa^\alpha$, where $\kappa$ is the noise covariance:
\begin{itemize}
\item $\alpha = 1$ (quantum): $g = \kappa$ (Fisher metric). Natural
  gradient dynamics.  Schr\"odinger-like evolution.
\item $\alpha = 1/2$ (efficient learning): intermediate.
  Adam/AdaBelief-like optimizers.
\item $\alpha = 0$ (classical): $g = I$ (flat metric). Ordinary gradient
  descent. Equilibration.
\end{itemize}
These regimes emerge from the power-law relationship between the metric
and the noise covariance, and they provide a richer structure than any
single-regime analysis.

\subsection{Implication for the bridge}
\label{sec:bridge-implication}

The Lovelock bridge, even if the continuum limit were somehow established,
would add no new physics to the Type~II framework.
Type~II satisfies all four Lovelock conditions \emph{from the start}:
general covariance is built into CGD (\cref{eq:CGD}),
second-order equations follow from the Einstein--Hilbert action
construction, the divergence-free condition follows from the Bianchi
identity on the derived Einstein tensor (\cref{eq:einstein-vacuum}), and
$D = 4$ is an architectural parameter.
Our bridge addresses a problem that no longer exists in Vanchurin's
current framework.

%----------------------------------------------------------------------
\section{The Mass-Fisher Metric Structure}
\label{sec:mass-fisher}
%----------------------------------------------------------------------

\subsection{Type~II metric decomposition}
\label{sec:metric-decomposition}

In the Type~II framework, the metric on trainable parameter space
decomposes as
\begin{equation}
\label{eq:metric-decomposition}
g_{\mu\nu}(\theta) = M_{\mu\nu}(\theta) + \beta\, F_{\mu\nu}(\theta)\,,
\end{equation}
where:
\begin{itemize}
\item $M_{\mu\nu}$ is the \textbf{mass tensor} (structural inertia):
  measures the cost of reconfiguring the observer's internal structure.
\item $F_{\mu\nu}$ is the \textbf{Fisher information metric}: measures
  the sensitivity of the observer's predictions to parameter changes.
\item $\beta = 1/(kT) > 0$ is the inverse temperature.
\end{itemize}

\begin{definition}[Structural inertia tensor]
\label{def:H1}
For an observer $O$ with internal hyperedges $E_O$, parameters
$\theta \in \Theta \subseteq \mathbb{R}^n$, and hyperedge configuration
functions $w_e : \Theta \to \mathbb{R}$, the structural inertia
tensor is
\begin{equation}
\label{eq:H1}
M^{(\mathrm{H1})}_{\mu\nu}(\theta) = \sum_{e \in E_O}
\frac{\partial w_e}{\partial \theta^\mu}\,
\frac{\partial w_e}{\partial \theta^\nu}
= (J^T J)_{\mu\nu}\,,
\end{equation}
where $J_{e,\mu} = \partial w_e / \partial \theta^\mu$ is the
$|E_O| \times n$ Jacobian matrix.
\end{definition}

$M^{(\mathrm{H1})}$ is symmetric, positive semi-definite (PSD), with
$\operatorname{rank}(M) \leq |E_O|$.
It depends on the observer's internal topology $(V_O, E_O)$, not on the
external environment.

\subsection{Hypothesis H1: $M = F^2$ for exponential family}
\label{sec:M-equals-F-squared}

\begin{theorem}[$M = F^2$ for exponential family models]
\label{thm:M-F-squared}
Let the observer's prediction model belong to the exponential family:
$p_\theta(x) = h(x) \exp(\theta^T T(x) - A(\theta))$, where $T(x)$ is
the sufficient statistic and $A(\theta)$ the log-partition function.
If $|E_O| = \dim(T)$ (the observer has one internal edge per sufficient
statistic component), with the natural bijection $e_i \leftrightarrow T_i$ (each internal
edge corresponds to one sufficient statistic component), and
the internal hyperedge configurations track the \emph{mean parameters}
($w_e(\theta) = \mathbb{E}_\theta[T_e] = \partial A / \partial\theta^e$),
then
\begin{equation}
\label{eq:M-F-squared}
M_{\mu\nu} = (F^2)_{\mu\nu} =
\sum_{\alpha} F_{\mu\alpha}\, F_{\alpha\nu}\,.
\end{equation}
\end{theorem}

\begin{proof}
For the exponential family, the Fisher matrix is
$F_{\mu\nu} = \partial^2 A / \partial\theta^\mu\, \partial\theta^\nu$.
Under the identification $w_e(\theta) = \partial A / \partial\theta^e$,
the Jacobian is
$J_{e\mu} = \partial w_e / \partial\theta^\mu
= \partial^2 A / \partial\theta^e\, \partial\theta^\mu = F_{e\mu}$.
Therefore $M = J^T J = F^T F = F^2$, where the last equality uses
the symmetry of~$F$.
This is a standard consequence of exponential family structure: the
three-line calculation $w_e = \partial A/\partial\theta^e \Rightarrow
\partial w_e/\partial\theta^a = F_{ea} \Rightarrow M = F^T F$ is exact
for canonical exponential families.
We have verified this numerically for the $2$-parameter Ising model
(with $w_e = \langle x_i x_j \rangle_\theta$, the expected
pairwise correlation) to relative error $< 10^{-8}$.
\end{proof}

\begin{remark}
The identity $J = F$ for exponential families is a well-known consequence
of the Hessian structure of the log-partition function~\cite{Amari2016}.
Our contribution is connecting this algebraic identity to Vanchurin's mass
tensor decomposition, yielding the specific relationship $M = F^2$ within
the Type~II metric framework.

The condition $w_e = \mathbb{E}_\theta[T_e]$ states that the observer's
internal structure tracks the expected sufficient statistics of its own
predictive model---a natural requirement for a Good Regulator.
Note the contrast: if the configurations are instead the natural
parameters themselves ($w_e = \theta^e$), then $M = I$ (identity),
not~$F^2$.
For general models outside the exponential family, $M$ and $F^2$
may differ; the deviation measures the departure from
exponential-family structure.
Whether $M = F^2$ extends beyond canonical exponential families
(e.g., to marginals such as restricted Boltzmann machines) remains
an open question requiring independent computation of the
Jacobian $J_{ea} = \partial w_e / \partial\theta^a$.
\end{remark}

\subsection{Spectral purity and the Good Regulator condition}
\label{sec:spectral-purity}

\begin{definition}[Structural Reflection Condition (SRC)]
\label{def:SRC}
An observer $O$ satisfies the SRC if for each internal hyperedge
$e \in E_O$, there exists a boundary observation $x_e$ and a constant
$c_e > 0$ such that
\[
\frac{\partial w_e}{\partial \theta^\mu} = c_e \cdot
\frac{\partial \log p_\theta(x_e)}{\partial \theta^\mu}
\quad \text{for all } \mu.
\]
\end{definition}

\begin{theorem}[SRC $\Rightarrow$ $M \propto F$]
\label{thm:SRC}
Let $O$ satisfy the SRC (\cref{def:SRC}), with the internal hyperedges in
bijection with boundary observations and
$c_e^2 = \kappa\, p(x_e)$ for all $e$.
(The SRC is a formalization of the Good Regulator
principle~\cite{ConantAshby1970,Virgo2025}---internal structure mirrors
external dynamics---not a derivation from it.)
Then
\begin{equation}
\label{eq:M-prop-F}
M_{\mu\nu} = \kappa\, F_{\mu\nu}\,,
\end{equation}
so that $g_{\mu\nu} = (\kappa + \beta)\, F_{\mu\nu}$:
the metric is conformally Fisher.
\end{theorem}

\begin{proof}
Under the SRC:
\begin{align*}
M_{\mu\nu} &= \sum_{e \in E_O} (\partial_\mu w_e)(\partial_\nu w_e) \\
&= \sum_e c_e^2\, (\partial_\mu \log p_\theta(x_e))\,
   (\partial_\nu \log p_\theta(x_e)) \\
&= \kappa \sum_e p(x_e)\, (\partial_\mu \log p_\theta(x_e))\,
   (\partial_\nu \log p_\theta(x_e)) \\
&= \kappa\, F_{\mu\nu}\,. \qedhere
\end{align*}
\end{proof}

\begin{corollary}[Deviation tensor]
\label{cor:deviation}
The deviation tensor $\Delta_{\mu\nu} = M_{\mu\nu} - \kappa\,F_{\mu\nu}$
measures the degree of imperfection in the observer's internal model.
$\Delta = 0$ for perfect Good Regulators satisfying SRC;
$\Delta \neq 0$ captures vestigial connections, incomplete mirroring,
and metabolic overhead.
\end{corollary}

\subsection{Observer complexity and regimes}
\label{sec:observer-complexity}

The mass-Fisher decomposition connects observer complexity to
Vanchurin's three regimes:

\begin{itemize}
\item \textbf{Simple observers} ($|E_O| = 0$): $M = 0$, so $g = \beta\,F$.
  This is the $\alpha = 1$ (quantum) regime.
  Dynamics are purely information-theoretic.

\item \textbf{Perfect Good Regulators}: $M = \kappa\,F$, so
  $g = (\kappa + \beta)\,F$.  Still effectively $\alpha = 1$ (rescaled
  Fisher geometry).

\item \textbf{Complex observers} ($|E_O| \gg n$, $M$ not proportional
  to~$F$): $M$ dominates at low temperature.
  As $\|M\| / (\beta \|F\|) \to \infty$, the effective $\alpha \to 0$
  (classical regime).
\end{itemize}

The transition from quantum ($\alpha = 1$) to classical ($\alpha = 0$)
is driven by the accumulation of internal structural complexity, not by
environmental decoherence.
This may provide a microscopic mechanism for the quantum-to-classical
transition within learning systems, complementary to Friston's
free-energy principle~\cite{Friston2010}, though this remains conjectural
(confidence: 40\%).

%----------------------------------------------------------------------
\section{Lorentzian Signature Mechanism}
\label{sec:lorentzian}
%----------------------------------------------------------------------

\subsection{PSD obstruction}
\label{sec:psd-obstruction}

\begin{theorem}[No Lorentzian from PSD mass tensor]
\label{thm:psd-obstruction}
If $M_{\mu\nu}$ is defined by \cref{eq:H1} and $\beta > 0$, then
$g = M + \beta\,F$ is positive semi-definite.
Lorentzian signature $(-,+,\ldots,+)$ is impossible.
\end{theorem}

\begin{proof}
$M = J^T J \geq 0$ (PSD by construction).
$F \geq 0$ (PSD by definition of Fisher metric).
$\beta > 0$.
Therefore $g = M + \beta\,F \geq 0$.
All eigenvalues are non-negative.
\end{proof}

This obstruction applies to all three standard hypotheses for $M$
(structural, accumulated Fisher, connectivity) and constitutes a
\emph{theorem}, not an open question.

\begin{remark}[Comprehensive empirical validation]
\label{rem:psd-phase-diagram}
An exhaustive phase diagram scanning 240~configurations---observer
sizes $m \in \{2, \ldots, 15\}$ (10~values), coupling strengths
$J \in [0.1, 2.0]$ (8~values), and three graph topologies (tree,
sparse, dense)---confirms that Lorentzian selection ($q = 1$)
achieves a win rate of exactly~$0\%$ when $M = F^2$.
In every configuration, $\beta_c = -1$ (no valid Lorentzian regime
exists) because $F^{-1/2} M\, F^{-1/2} = F$, which is positive
definite with no negative eigenvalues.
The obstruction is structural and independent of observer size,
coupling strength, and graph topology: it holds universally for
canonical exponential families where $M = F^2$.

This validates the necessity of the signed-edge construction~H1$'$
(\cref{sec:H1-prime}) or Vanchurin's non-principal square root for
obtaining Lorentzian signature.
A contrast experiment confirms that H1$'$ does bypass the PSD
obstruction; see \cref{rem:signed-edge-phase-diagram}.
\end{remark}

\subsection{Modified construction H1$'$: signed edge weights}
\label{sec:H1-prime}

To obtain Lorentzian signature, we introduce signed edge weights
$s_e \in \{+1, -1\}$:

\begin{definition}[Modified structural inertia tensor]
\label{def:H1-prime}
\begin{equation}
\label{eq:H1-prime}
M^{(\mathrm{H1'})}_{\mu\nu}(\theta) = \sum_{e \in E_O}
s_e\, \frac{\partial w_e}{\partial \theta^\mu}\,
\frac{\partial w_e}{\partial \theta^\nu}\,.
\end{equation}
The edge set partitions into spacelike ($E^+ = \{e : s_e = +1\}$,
cardinality~$p$) and timelike ($E^- = \{e : s_e = -1\}$,
cardinality~$q$), with $M^{(\mathrm{H1'})} = M^+ - M^-$.
Both $M^+$ and $M^-$ are PSD (Gram matrices).
\end{definition}

Physical motivation: the sign assignment $s_e$ may correspond to the
causal character of the hyperedge---edges connecting causally related
vertices carry $s_e = -1$ (timelike), while edges connecting causally
disconnected vertices carry $s_e = +1$ (spacelike).
This assignment is currently imposed rather than derived; its
physical origin is the most important open problem in this analysis
(see \cref{sec:open-problems}).

\begin{remark}[H1$'$ contrast experiment: signed-edge phase diagram]
\label{rem:signed-edge-phase-diagram}
A contrast experiment repeating the 240-configuration phase diagram
of \cref{rem:psd-phase-diagram} with the H1$'$ signed mass tensor
(using Fiedler vector sign assignment) reveals a qualitatively
different regime.
Under H1$'$, the critical coupling $\beta_c > 0$ for most
non-tree configurations (e.g., $\beta_c \approx 1.08$--$1.38$
for dense graphs at weak coupling $J \leq 0.3$), confirming that
the signed-edge construction \emph{does} bypass the PSD obstruction
and creates a valid Lorentzian regime.
However, the spectral gap weighting $W(q{=}1)$ equals the maximum
$W_{\mathrm{max}}$ across all signatures---a degeneracy in which
$q = 1$ ties with higher-$q$ signatures rather than strictly
dominating.
This degeneracy is an \emph{exact algebraic identity}: the signature
matrix $\Sigma = \operatorname{diag}(s_1, \ldots, s_m)$ with
$s_i \in \{+1, -1\}$ satisfies $\Sigma^{-1} = \Sigma$, so the
conjugation $\Sigma\, A\, \Sigma = \Sigma\, A\, \Sigma^{-1}$ is a
similarity transformation preserving all eigenvalues of~$A$ and hence
$\beta_c$, $L_{\mathrm{gap}}$, and $W(q)$ for every~$q$.
No mass tensor of the form $M = S\, f(F)\, S$ can break this
degeneracy; signature selection must instead rely on the eigenvalue
structure of~$A = F^{1/2}\, S\, F^{1/2}$ directly
(\cref{cor:signature-rigidity}).
The Lorentzian regime is \emph{accessible} under H1$'$ but not
\emph{preferentially selected} by the spectral gap mechanism alone,
suggesting that an additional physical principle is required to
break the degeneracy in favor of $q = 1$.
\end{remark}

\subsection{FSF factorization for exponential families}
\label{sec:FSF-factorization}

For canonical exponential family observers, the signed mass tensor admits
a remarkably simple factorization through the Fisher matrix.

\begin{theorem}[Signed mass tensor factorization]
\label{thm:FSF}
For a canonical exponential family with $m$-dimensional natural parameters
$\theta$, Fisher information matrix~$F$, and sign assignment
$S = \operatorname{diag}(s_1, \ldots, s_m)$ with $s_e \in \{+1, -1\}$:
\begin{equation}
\label{eq:FSF}
M^{(\mathrm{H1'})} = F\,S\,F\,.
\end{equation}
\end{theorem}

\begin{proof}
By \cref{thm:M-F-squared}, the Jacobian of the mean-parameter map is
the Fisher matrix: $\partial w_e / \partial \theta^a = F_{ea}$.
Therefore
\begin{equation}
M^{(\mathrm{H1'})}_{ab}
= \sum_e s_e \frac{\partial w_e}{\partial \theta^a}
\frac{\partial w_e}{\partial \theta^b}
= \sum_e F_{ae}\, S_{ee}\, F_{eb}
= (F\,S\,F)_{ab}\,.
\end{equation}
\end{proof}

\begin{corollary}[Signature rigidity]
\label{cor:signature-rigidity}
The Fisher-weighted operator
$A = F^{-1/2}\, M^{(\mathrm{H1'})}\, F^{-1/2} = F^{1/2}\, S\, F^{1/2}$
has exactly $q = |\{e : s_e = -1\}|$ negative eigenvalues, by
Sylvester's law of inertia.
Pure Lorentzian signature (exactly one negative eigenvalue of $g(\beta)$
for all $\beta \in (0, \beta_c)$) therefore requires exactly $q = 1$.
\end{corollary}

This result eliminates free parameters from the Lorentzian mechanism:
once the Fisher matrix~$F$ (determined by the observer's graph topology)
and the sign assignment~$S$ are specified, the entire metric structure
$g(\beta) = F\,S\,F + \beta\,F$ is uniquely determined.

\subsection{Critical beta theorem}
\label{sec:beta-c-theorem}

\begin{theorem}[Critical $\beta$ for Lorentzian transition]
\label{thm:beta-c}
Let $g(\beta) = M^{(\mathrm{H1'})} + \beta\,F$ where
$M^{(\mathrm{H1'})}$ has at least one negative eigenvalue and $F$ is
positive definite.
Define $A = F^{-1/2}\, M^{(\mathrm{H1'})}\, F^{-1/2}$ with
eigendecomposition $A = P D P^T$, $D = \operatorname{diag}(d_1, \ldots, d_n)$,
$d_1 \leq d_2 \leq \cdots \leq d_n$.
Then:
\begin{equation}
\label{eq:beta-c}
\beta_c = -d_1 \qquad (\text{if } d_1 < 0)\,.
\end{equation}
Furthermore:
\begin{enumerate}
\item[\emph{(a)}] $0 < \beta_c < \infty$.
\item[\emph{(b)}] For $0 < \beta < \beta_c$: at least one eigenvalue of
  $g(\beta)$ is negative (Lorentzian or stronger indefinite signature).
\item[\emph{(c)}] For $\beta > \beta_c$: $g(\beta)$ is positive definite
  (Riemannian).
\item[\emph{(d)}] At $\beta = \beta_c$: $g(\beta_c)$ is degenerate
  (at least one zero eigenvalue).
\end{enumerate}
\end{theorem}

\begin{figure}[htbp]
\centering
\includegraphics[width=0.8\textwidth]{Fig1_eigenvalue_trajectories.png}
\caption{Eigenvalue trajectories of $A = F^{-1/2} M^{(\mathrm{H1'})} F^{-1/2}$
as a function of inverse temperature $\beta$. The most negative eigenvalue
$d_1(\beta)$ determines the critical value $\beta_c = -d_1$ at which the
metric $g = M + \beta F$ transitions from indefinite (Lorentzian or stronger)
to positive definite (Riemannian). Eigenvalues shown for observer topology O3
(path graph, 2 timelike edges) on a 2-parameter Gaussian model.}
\label{fig:eigenvalue-trajectories}
\end{figure}

\begin{proof}
Define $A = F^{-1/2}\, M^{(\mathrm{H1'})}\, F^{-1/2}$ (well-defined
since $F$ is positive definite).
In the $F$-orthonormal basis $y = F^{1/2}\,\theta$, the metric becomes
$y^T (A + \beta\, I)\, y$.
The eigenvalues of $A + \beta\, I$ are $d_i + \beta$.

\emph{(a)}
Since $d_1 < 0$, at $\beta = 0$ the smallest eigenvalue $d_1 < 0$.
As $\beta$ increases, all eigenvalues $d_i + \beta$ increase strictly.
The smallest eigenvalue crosses zero at $\beta_c = -d_1 > 0$.
For the upper bound: for any unit vector~$u$,
$u^T g(\beta)\, u \geq -\|M^-\| + \beta\,\lambda_{\min}(F)$,
which is positive when $\beta > \|M^-\| / \lambda_{\min}(F) < \infty$.

\emph{(b)}
For $\beta < \beta_c = -d_1$, we have $d_1 + \beta < 0$, so $g(\beta)$
has at least one negative eigenvalue.

\emph{(c)}
For $\beta > \beta_c$, all $d_i + \beta > 0$ (since $d_1$ is the most
negative), so $g(\beta)$ is positive definite.

\emph{(d)}
By continuity, $d_1 + \beta_c = 0$.
\end{proof}

The eigenvalue trajectories and signature transitions are illustrated
in \cref{fig:eigenvalue-trajectories}.

\begin{remark}[Prior art]
\label{rem:beta-c-prior-art}
The critical value $\beta_c = -d_1$ follows from standard results on
simultaneous diagonalization of symmetric
matrices~\cite{HornJohnson2013}.
The physical content lies in the observer-topology-dependent
characterization of the Lorentzian--Riemannian transition within the
Type~II framework.
We note that signature change in general relativity has been studied
by Hayward~\cite{Hayward1992} and others in the context of classical
spacetime; our result concerns signature transitions in the parameter
space of learning systems, which is a different physical setting.
\end{remark}

\begin{corollary}[Signature counting]
\label{cor:signature-counting}
The number of negative eigenvalues of $g(\beta)$ equals
$\#\{i : d_i < -\beta\}$.
In particular, if $d_1 < 0$ and $d_2 \geq 0$ (only one negative
eigenvalue of~$A$), then $g$ has Lorentzian signature for all
$\beta \in (0,\, \beta_c)$.
If $d_2 < 0$ as well, define $\beta_{c,2} = -d_2$.
Then $g$ has Lorentzian signature (exactly one negative eigenvalue)
for $\beta \in (\beta_{c,2},\, \beta_c)$, and two or more negative
eigenvalues for $\beta \in (0,\, \beta_{c,2})$.
\end{corollary}

\begin{remark}[Physical interpretation]
Since $\beta = 1/(kT)$, the Lorentzian regime corresponds to
\emph{high} temperature (small~$\beta$) and the Riemannian regime to
\emph{low} temperature (large~$\beta$).
The Lorentzian threshold $\beta_c$ is a property of the observer
(determined by $M$ and~$F$), not of the thermal environment.
For exponential family observers, the FSF factorization
(\cref{thm:FSF}) implies that $A = F^{1/2}\, S\, F^{1/2}$, so
$\beta_c$ is determined entirely by $F$ and~$S$---no free parameters
remain once the observer's graph topology and sign assignment are fixed.
\end{remark}

\begin{figure}[htbp]
\centering
\includegraphics[width=0.8\textwidth]{Fig2_phase_diagram.png}
\caption{Phase diagram showing Lorentzian and Riemannian regimes in the
$(\beta, \|M^-\|)$ plane. The critical curve $\beta_c = -d_1(M, F)$
separates indefinite signature (below, high temperature) from positive
definite signature (above, low temperature). Observers with higher structural
complexity (larger $\|M^-\|$, more timelike edges) have larger Lorentzian
regimes. Data points show six observer topologies (O1--O6) from
\cref{tab:observers}.}
\label{fig:phase-diagram}
\end{figure}

\subsection{Topology dependence}
\label{sec:topology}

\begin{theorem}[Minimum complexity for Lorentzian signature]
\label{thm:min-complexity}
\begin{enumerate}
\item[\emph{(a)}] An observer with $|E_O| = 0$ has $M = 0$, so
  $g = \beta\,F$ (Riemannian).  No Lorentzian signature is possible.

\item[\emph{(b)}] An observer with exactly one timelike edge
  ($|E^-| = 1$, $|E^+| = 0$) in parameter dimension $n \geq 2$ can
  have Lorentzian signature for $\beta < \beta_c$.
\end{enumerate}
The minimum complexity for Lorentzian signature is therefore:
at least one timelike internal edge, and $\dim(\Theta) \geq 2$.
\end{theorem}

\begin{proof}
Part~(a) is immediate from $M = 0$.
For part~(b), a single timelike edge with configuration $w_e$ gives
$M^{(\mathrm{H1'})} = -(\nabla w_e)(\nabla w_e)^T$, a negative
semi-definite rank-1 matrix.
In the direction of $\nabla w_e$:
$g(\nabla w_e, \nabla w_e) = -\|\nabla w_e\|^4 + \beta\,
(\nabla w_e)^T F\, (\nabla w_e)$,
which is negative for sufficiently small~$\beta$.
In orthogonal directions $\xi$ ($\xi \cdot \nabla w_e = 0$):
$g(\xi, \xi) = \beta\, \xi^T F\, \xi > 0$.
So $g$ has signature $(n-1, 0, 1) =$ Lorentzian for $\beta < \beta_c$.
\end{proof}

\begin{proposition}[Edge monotonicity]
\label{prop:edge-monotonicity}
Adding a timelike edge to an observer expands the Lorentzian regime:
\begin{equation}
\label{eq:edge-monotonicity}
\beta_c(E_O \cup \{e_{\mathrm{new}},\, s = -1\}) \;\geq\; \beta_c(E_O)\,.
\end{equation}
Adding a spacelike edge contracts it:
$\beta_c(E_O \cup \{e_{\mathrm{new}},\, s = +1\}) \leq \beta_c(E_O)$.
\end{proposition}

\begin{figure}[htbp]
\centering
\includegraphics[width=0.75\textwidth]{Fig5_edge_addition_monotonicity.png}
\caption{Edge addition monotonicity (\cref{prop:edge-monotonicity}). Adding
timelike edges (causal connections, $s=-1$) monotonically increases $\beta_c$,
expanding the high-temperature Lorentzian regime. Adding spacelike edges
($s=+1$) decreases $\beta_c$, contracting the Lorentzian regime. Demonstrated
for observer topologies O1--O6 on a 2-parameter Gaussian model. Error bars
show numerical precision ($< 10^{-3}$).}
\label{fig:edge-monotonicity}
\end{figure}

\begin{proof}
Adding a timelike edge contributes $-(\nabla w_{e_{\mathrm{new}}})
(\nabla w_{e_{\mathrm{new}}})^T$ to $M^{(\mathrm{H1'})}$, a negative
semi-definite rank-1 perturbation.
By the Weyl interlacing inequalities, the most negative eigenvalue of
$M^{(\mathrm{H1'})}$ can only become more negative (or stay the same),
so $\beta_c = -d_1$ increases or stays constant.
The spacelike case is symmetric.
\end{proof}

\begin{remark}
\Cref{prop:edge-monotonicity} holds when $F$ is fixed independently
of the edge addition.
In self-consistent models where the observer's topology determines
both $M$ and $F$ (e.g., Ising models on the observer graph),
adding edges changes $F$, and monotonicity can be violated.
See the computational notes accompanying this paper for an explicit
example.
\end{remark}

We have verified these results numerically for six observer topologies
(O1--O6) on a 2-parameter Gaussian model, as shown in \cref{tab:observers}.
High-dimensional computational analysis ($n = 2$ to $10$, 3500 configurations)
confirms that Lorentzian signature selection remains robust, with 98\% of
random observer topologies producing exactly one timelike direction.

\begin{table}[ht]
\centering
\caption{Numerical verification of $\beta_c$ for six observer topologies
on a 2-parameter Gaussian model ($\theta = (\mu, \log\sigma)$).
Sign assignment: vertices connected by an edge in $G$ $\to$ timelike ($s = -1$),
vertices not connected $\to$ spacelike ($s = +1$).}
\label{tab:observers}
\begin{tabular}{@{}lccccr@{}}
\toprule
Observer & $|V_O|$ & $|E^-|$ & $|E^+|$ & Lorentzian? & $\beta_c$ \\
\midrule
O1 (single vertex)   & 1 & 0 & 0 & No  & $0$ \\
O2 (pair, 1 edge)    & 2 & 1 & 0 & Yes & $1.15$ \\
O3 (path, 2 edges)   & 3 & 2 & 0 & Yes & $1.89$ \\
O4 (triangle, $K_3$) & 3 & 2 & 1 & Yes & $2.68$ \\
O5 (path, 3 edges)   & 4 & 3 & 0 & Yes & $2.41$ \\
O6 (complete, $K_4$) & 4 & 4 & 2 & Yes & $2.02$ \\
\bottomrule
\end{tabular}
\end{table}

\subsection{Comparison with Vanchurin's non-principal square root}
\label{sec:comparison-vanchurin}

Vanchurin~\cite{Vanchurin2025} obtains Lorentzian signature from PSD
gradient covariance via a non-principal square root:
\begin{align}
\label{eq:vanchurin-sqrt}
C_{\mu\nu} &= \langle p_\mu\, p_\nu \rangle = Q\,\Lambda\,Q^T
\quad (\text{PSD eigendecomposition}), \\
g_{\mu\nu} &= (Q\,\Sigma\,\Lambda^{1/2}\,Q^T)_{\mu\nu}\,,
\quad \Sigma = \operatorname{diag}(\sigma_0, \sigma_1, \ldots)\,,
\quad \sigma_i = \pm 1. \nonumber
\end{align}
Choosing $\sigma_0 = -1$ (one sign flip) produces Lorentzian~$g$.

\begin{theorem}[Non-equivalence of constructions]
\label{thm:non-equivalence}
The H1$'$ signed-edge construction (\cref{def:H1-prime}) and
Vanchurin's non-principal square root (\cref{eq:vanchurin-sqrt}) are
\emph{not equivalent} in general:
\begin{enumerate}
\item[\emph{(a)}] Vanchurin's construction is a \emph{global} operation:
  signs are assigned in the eigenbasis of the total gradient covariance~$C$.
\item[\emph{(b)}] H1$'$ is a \emph{local} operation:
  signs are assigned to individual structural edges before summation.
\item[\emph{(c)}] Neither construction subsumes the other:
  there exist sign matrices $\Sigma$ unrealizable by any H1$'$ assignment,
  and there exist H1$'$ mass tensors not expressible as non-principal
  square roots of any covariance.
\end{enumerate}
\end{theorem}

\begin{proof}
Parts~(a) and (b) are definitional.
For part~(c), we give an explicit counterexample with $n = 2$,
$|E_O| = 3$.
Let $C = \operatorname{diag}(4, 1)$ (PSD covariance).
Vanchurin's construction with $\Sigma = \operatorname{diag}(-1, +1)$
produces $g^{(\mathrm{V})} = \operatorname{diag}(-2, 1)$.
Now consider all H1$'$ metrics with three rank-1 terms
$M^{(\mathrm{H1'})} = \sum_{e=1}^{3} s_e\, v_e v_e^T$.
For $M^{(\mathrm{H1'})}$ to equal $g^{(\mathrm{V})}$
(up to a Fisher shift), we need the $(1,1)$ entry to be negative
and the $(2,2)$ entry to be positive.
Since each $v_e v_e^T$ has non-negative diagonal entries, the only
way to make the $(1,1)$ entry negative is via timelike edges
($s_e = -1$) with $v_e$ having nonzero first component.
But such edges also contribute negatively to the $(2,2)$ entry
(unless $v_e = (a, 0)^T$), coupling the two signs.
One can verify numerically that no choice of three vectors
$v_1, v_2, v_3 \in \mathbb{R}^2$ and signs
$s_1, s_2, s_3 \in \{+1, -1\}$ produces
$M^{(\mathrm{H1'})} = \operatorname{diag}(-2, 1)$ exactly,
establishing that some Vanchurin metrics are not H1$'$-realizable.
Conversely, H1$'$ with non-aligned edge directions produces off-diagonal
mass tensor entries that Vanchurin's diagonal construction cannot achieve.
\end{proof}

\begin{remark}
H1$'$ and Vanchurin's construction are mathematically distinct
\emph{formulations} that may or may not produce equivalent observables.
H1$'$ may be interpreted as a candidate microscopic mechanism
whose macroscopic limit could coincide with Vanchurin's non-principal
square root; testing this requires deriving predictions that differ
between the two approaches, which remains an open problem.
Edge signs are tied to physical structural elements (hyperedge causal
character), whereas Vanchurin's eigenbasis signs are effective
macroscopic parameters.
This interpretation is conjectural (confidence: 50\%).
\end{remark}

\subsection{Persistence under learning}
\label{sec:persistence}

\begin{theorem}[Rate of change of $\beta_c$]
\label{thm:persistence}
Let $\beta_c(t) = \beta_c(\theta(t))$ along a natural gradient
trajectory $\dot{\theta}^\mu = -\gamma\, g^{\mu\nu}\, \partial_\nu H$.
Then
\begin{equation}
\label{eq:dbeta-dt}
\frac{d\beta_c}{dt} = -\frac{dd_1}{dt}
= -v_1^T\, \frac{dA}{dt}\, v_1\,,
\end{equation}
where $v_1(t)$ is the unit eigenvector of
$A(t) = F^{-1/2}\, M^{(\mathrm{H1'})}\, F^{-1/2}$ corresponding to its
most negative eigenvalue $d_1(t)$, and $dA/dt$ depends on the learning
direction $\dot{\theta}$ via derivatives of both $M$ and~$F$.
\end{theorem}

\begin{proof}
Follows from \cref{thm:beta-c} ($\beta_c = -d_1$), the chain rule, and
the Hellmann--Feynman theorem for the perturbation of eigenvalues of a
continuously parametrized symmetric matrix.
\end{proof}

\begin{proposition}[Stability near local minima]
\label{prop:stability}
Near a local minimum $\theta^*$ of the loss $H$, the learning velocity
$\dot{\theta} \to 0$, so $d\beta_c/dt \to 0$.
An observer in the Lorentzian regime at $\theta^*$ remains Lorentzian for
a time of order $1 / |\dot{\theta}|$ (the convergence timescale).
\end{proposition}

\subsection{Computational verification of the signature transition}
\label{sec:computational-verification}

We tested the $\beta_c$ prediction (\cref{thm:beta-c}) on Ising/Boltzmann
machines with exact enumeration across seven observer topologies
(3--5~vertices, 2--10~edges) and five coupling strengths each, for a
total of 35~test cases.
The signed mass tensor $M^{(\mathrm{H1'})}$ was constructed using the
adjacency-based sign rule (adjacent vertices $\to$ timelike,
non-adjacent $\to$ spacelike).

Of 35~cases where $M^{(\mathrm{H1'})}$ is indefinite and $\beta_c$ is
predicted, \textbf{32 match with relative error $< 1\%$}; the
average relative error across all tests is $2.15 \times 10^{-2}$.
The three mismatches occur at strong coupling ($J \geq 1.0$) on
complete graphs ($K_4$, $K_5$), where higher-order corrections to the
quadratic approximation become significant.

Three additional findings emerged from the computational experiments:
\begin{enumerate}
\item \textbf{Specific heat peak does not coincide with $\beta_c$.}
  The thermodynamic phase transition (specific heat maximum) occurs at
  $\beta_{\mathrm{eff}}$ values systematically different from $\beta_c$.
  This confirms that $\beta_c$ characterizes a \emph{geometric}
  transition (metric signature change), not a thermodynamic one.
  See \cref{fig:specific-heat} for the systematic deviation between
  $\beta_c$ (geometric transition) and $\beta_{\mathrm{eff}}$
  (thermodynamic peak).

\item \textbf{Learning dynamics change qualitatively at $\beta_c$.}
  Natural gradient descent with metric $g = M^{(\mathrm{H1'})} + \beta F$
  exhibits three distinct regimes: unstable amplification along the
  timelike direction for $\beta < \beta_c$ (Lorentzian), singular
  behavior at $\beta = \beta_c$ (degenerate), and smooth convergent
  descent for $\beta > \beta_c$ (Riemannian).
  See \cref{fig:learning-dynamics} for trajectory divergence at the
  critical transition.

\item \textbf{Multi-eigenvalue transition sequence
  (\cref{cor:signature-counting}).}
  For observers with multiple negative eigenvalues of the
  Fisher-weighted mass tensor $A = F^{-1/2}\,M^{(\mathrm{H1'})}\,F^{-1/2}$,
  the predicted sequence of signature transitions
  (indefinite $\to$ Lorentzian $\to$ Riemannian) is verified exactly.
\end{enumerate}

Full results, including eigenvalue trajectories, phase diagrams,
and learning dynamics traces, are available in the code repository.

\subsection{Spectral gap selection of Lorentzian signature}
\label{sec:spectral-gap-selection}

The preceding results establish how the Lorentzian--Riemannian transition
depends on observer topology, but do not address \emph{which} sign
assignment~$q$ (number of timelike edges) is physically selected.
We introduce a spectral gap weighting that provides a conditional answer.

\begin{definition}[Spectral gap weighting]
\label{def:spectral-gap-weighting}
For a positive definite Fisher matrix~$F$ and sign assignment~$S$ with
$q$ negative entries, define:
\begin{align}
A(S) &= F^{1/2}\, S\, F^{1/2}\,, \quad
\text{eigenvalues } d_1 \leq d_2 \leq \cdots \leq d_m\,, \\
\beta_c(q) &= \max_{S:\,|S^-|=q} (-d_1(S))\,, \\
L_{\mathrm{gap}}(q) &= \frac{d_2 - d_1}{|d_1|}
\bigg|_{\text{at } S^*}\,, \\
W(q) &= \beta_c(q) \cdot L_{\mathrm{gap}}(q)\,.
\end{align}
The spectral gap ratio $L_{\mathrm{gap}}$ measures how sharply the
most negative eigenvalue is separated from the bulk spectrum.
Lorentzian signature ($q = 1$) corresponds to a single, spectrally
isolated timelike direction.
\end{definition}

\begin{theorem}[Spectral gap lower bound for $q = 1$]
\label{thm:Lgap-bound}
For any positive definite $F$ with eigenvalues
$\lambda_1 \geq \cdots \geq \lambda_m > 0$, the sign assignment with
$q = 1$ satisfies $L_{\mathrm{gap}}(q = 1) \geq 1$.
For $q \geq 2$, $L_{\mathrm{gap}}$ can equal zero.
\end{theorem}

\begin{proof}
For $q = 1$ with the $k$-th edge negative,
$A = F - 2\,f_k f_k^T$ where $f_k = F^{1/2} e_k$.
This is a rank-1 perturbation of the positive definite matrix~$F$.
By the Cauchy interlacing theorem, the eigenvalues $\mu_1 \geq \cdots
\geq \mu_m$ of $A$ satisfy $\mu_{m-1} \geq \lambda_m > 0$.
Therefore $d_2 = \mu_{m-1} > 0$ while $d_1 = \mu_m < 0$,
giving $L_{\mathrm{gap}} = (d_2 - d_1)/|d_1| \geq \lambda_m/|d_1| + 1
\geq 1$.

For $q = 2$ with $F = I$: $A = I - 2(e_j e_j^T + e_k e_k^T)$ has
eigenvalues $+1$ ($m - 2$ times) and $-1$ (twice).
The two negative eigenvalues are degenerate, so
$L_{\mathrm{gap}} = 0$.
\end{proof}

This theorem has a clean physical interpretation: $q = 1$ produces a
\emph{single} spectrally isolated timelike direction (large gap), while
$q \geq 2$ produces \emph{degenerate} timelike directions that cannot
be distinguished (zero or small gap).

\begin{theorem}[Tree Fisher identity]
\label{thm:tree-fisher}
For an Ising model on a tree graph~$G$ (connected, acyclic) with
uniform coupling~$J$, the Fisher information matrix in the edge
parameterization is
\begin{equation}
\label{eq:tree-fisher}
F = \operatorname{sech}^2(J)\, I_m\,,
\end{equation}
where $I_m$ is the $m \times m$ identity matrix.
\end{theorem}

\begin{proof}
On a tree with $n$ vertices and $m = n - 1$ edges, fix a root vertex
$v_0$.
Each spin configuration $(s_1, \ldots, s_n)$ is uniquely determined by
the reference spin $s_{v_0}$ and the edge variables
$\sigma_e = s_i s_j$ for each edge~$e = (i,j)$.
The Hamiltonian $H = -J \sum_e \sigma_e$ depends only on the edge
variables, and since there are no cycles, there are no algebraic
constraints between distinct $\sigma_e$.
The Boltzmann distribution therefore factorizes:
$P(\sigma_{e_1}, \ldots, \sigma_{e_{m}})
= \prod_e P(\sigma_e)$,
with each $\sigma_e$ an independent Bernoulli variable
($P(\sigma_e = +1) = (1 + \tanh J)/2$).
The Fisher matrix $F_{ab} = \operatorname{Cov}(\sigma_a, \sigma_b)
= \delta_{ab}\, \operatorname{Var}(\sigma_e)
= \delta_{ab}\, \operatorname{sech}^2(J)$.
\end{proof}

\begin{corollary}[Non-uniform Tree Fisher identity]
\label{cor:tree-fisher-nonuniform}
For an Ising model on a tree with edge-dependent couplings
$J_e > 0$,
\begin{equation}
\label{eq:tree-fisher-nonuniform}
F = \operatorname{diag}\bigl(
  \operatorname{sech}^2(J_{e_1}),\;\ldots,\;
  \operatorname{sech}^2(J_{e_m})
\bigr).
\end{equation}
\end{corollary}

\begin{proof}
The proof of \cref{thm:tree-fisher} carries through with
$H = -\sum_e J_e \sigma_e$: the factorization
$P(\sigma_{e_1}, \ldots, \sigma_{e_m}) = \prod_e P(\sigma_e)$
still holds (no cycle constraints), and each marginal gives
$\operatorname{Var}(\sigma_e) = \operatorname{sech}^2(J_e)$.
Verified numerically for 49~non-uniform configurations
(paths~$P_4$--$P_6$, stars~$S_4$--$S_6$, random couplings
$J_e \in [0.1, 2.0]$; all match to machine precision).
\end{proof}

\begin{corollary}[Absolute Lorentzian dominance on trees]
\label{cor:tree-lorentzian}
For $F = c\,I_m$ with $c > 0$ and $m \geq 2$:
$W(q = 1) = 2c > 0 = W(q \geq 2)$.
Therefore $q = 1$ is the \textbf{unique} maximizer of $W$ on tree
observer graphs with uniform coupling, with infinite margin.
\end{corollary}

\begin{proof}
For $F = c\,I$, every $A(S) = c\,S$.
For $q = 1$: eigenvalues $+c$ ($m - 1$ times) and $-c$ (once),
$\beta_c = c$, $L_{\mathrm{gap}} = 2$, $W = 2c$.
For $q \geq 2$: eigenvalue $-c$ has multiplicity $\geq 2$
(degenerate), $L_{\mathrm{gap}} = 0$, $W = 0$.
\end{proof}

\begin{theorem}[Lorentzian dominance for diagonal F]
\label{thm:diagonal-lorentzian}
For any positive definite diagonal matrix
$F = \operatorname{diag}(d_1, \ldots, d_m)$ with
$d_1 \geq \cdots \geq d_m > 0$ and $m \geq 2$:
\begin{equation}
\label{eq:diagonal-W-comparison}
W(q = 1) = d_1 + d_m > d_1 - d_2 = \max_{q \geq 2}\, W(q)\,.
\end{equation}
In particular, $W(q = 1) - \max_{q \geq 2} W(q) = d_2 + d_m > 0$.
\end{theorem}

\begin{proof}
For diagonal~$F$, the signed kernel
$A(S) = F^{1/2}\,S\,F^{1/2} = \operatorname{diag}(s_1 d_1,
\ldots, s_m d_m)$.

\emph{Case $q = 1$}: To maximise $\beta_c = \max_k d_k$,
negate the largest entry ($k = 1$).
Eigenvalues: $-d_1, d_2, \ldots, d_m$.
Then $d_1^{(\mathrm{eig})} = -d_1$,
$d_2^{(\mathrm{eig})} = d_m$ (smallest positive),
$L_{\mathrm{gap}} = (d_m + d_1)/d_1$,
$W(1) = d_1 \cdot (d_m + d_1)/d_1 = d_1 + d_m$.

\emph{Case $q \geq 2$}: To maximise $\beta_c$, negate the
$q$ largest entries (indices $1, \ldots, q$).
Eigenvalues: $-d_1, -d_2, \ldots, -d_q, d_{q+1}, \ldots, d_m$.
The two most negative eigenvalues are $-d_1$ and~$-d_2$
(since $d_1 \geq d_2$), giving
$L_{\mathrm{gap}} = (d_1 - d_2)/d_1$,
$W(q) = d_1 \cdot (d_1 - d_2)/d_1 = d_1 - d_2$.
Note $W(q)$ is independent of~$q$ for $q \geq 2$.

The margin $W(1) - W(q \geq 2) = d_2 + d_m > 0$ since
$d_2, d_m > 0$.
\end{proof}

\begin{corollary}[Perturbation stability of Lorentzian dominance]
\label{cor:lorentzian-perturbation}
For a positive definite diagonal $D$ with eigenvalues
$d_1 \geq \cdots \geq d_m > 0$ and any symmetric perturbation~$P$
with $\|P\|_{\mathrm{op}} = 1$, there exists
$\delta_0 = \delta_0(D) > 0$ such that
$W(q = 1;\,D + \varepsilon P) > W(q;\,D + \varepsilon P)$ for
all $q \geq 2$ whenever $0 \leq \varepsilon < \delta_0$.
Numerical perturbation analysis shows that Lorentzian dominance
persists until $\|F - \operatorname{diag}(F)\|_{\text{op}} /
\|\operatorname{diag}(F)\|_{\text{op}} \approx 0.9$
for typical random perturbations (50~configurations).
\end{corollary}

\begin{proof}
By \cref{thm:diagonal-lorentzian}, $W(1; D) - \max_{q \geq 2}
W(q; D) = d_2 + d_m > 0$.
Since eigenvalues, and hence $\beta_c$ and $L_{\mathrm{gap}}$,
depend continuously on the matrix entries, $W(q; D + \varepsilon P)$
is a continuous function of~$\varepsilon$.
At $\varepsilon = 0$ the gap is $d_2 + d_m > 0$; by continuity,
$\delta_0 > 0$ exists such that the gap remains positive
for $\varepsilon < \delta_0$.
\end{proof}

For graphs with cycles, the off-diagonal corrections to~$F$ are
controlled by the girth~$g$ (shortest cycle length):

\begin{theorem}[Near-diagonal Fisher for sparse graphs]
\label{thm:near-diagonal-fisher}
For an Ising model on a connected graph with girth~$g \geq 3$,
maximum degree~$\Delta$, and uniform coupling~$J > 0$, the Fisher
information matrix satisfies:
\begin{equation}
\label{eq:near-diagonal-bound}
\frac{\|F - \operatorname{diag}(F)\|_{\text{op}}}
     {\|\operatorname{diag}(F)\|_{\text{op}}}
\leq C(\Delta) \cdot \tanh^{g-2}(J)\,,
\end{equation}
where the exponent $g - 2$ is exact (from transfer matrix analysis)
and the prefactor is
\begin{equation}
\label{eq:C-derived}
C(\Delta) = 2(\Delta - 1)\,,
\end{equation}
a universal constant equal to the maximum number of adjacent edges
(derived from the Dobrushin uniqueness condition and Gershgorin's
theorem; see~\cite{Dobrushin1985}).
On cycle graphs, the adjacent-edge covariance is given exactly by:
\begin{equation}
\label{eq:adjacent-edge-cov}
\operatorname{Cov}(\sigma_e, \sigma_f)
= \frac{\tanh^{g-2}(J) \cdot \operatorname{sech}^4(J)}
       {\bigl(1 + \tanh^g(J)\bigr)^2}\,,
\end{equation}
verified to machine precision for all~$g = 3, \ldots, 12$ and
$J \in [0.1, 1.0]$.
Trees ($g = \infty$) achieve exact diagonality
(\cref{thm:tree-fisher}).
Sparse graphs (large~$g$) have near-diagonal Fisher matrices, so the
spectral gap analysis of
\cref{thm:Lgap-bound,cor:tree-lorentzian} applies perturbatively.
\end{theorem}

\begin{proof}
On trees, edge variables $\sigma_e = s_i s_j$ are independent
(no algebraic constraints from cycles), giving diagonal~$F$
(\cref{thm:tree-fisher}).
For graphs with finite girth~$g$, correlations between edge variables
propagate through cycles. Consider two adjacent
edges $e = (i,j)$ and $f = (j,k)$ sharing vertex~$j$ on a
cycle~$C_g$. The product $\sigma_e \sigma_f = s_i s_k$ connects
vertices at graph distance~$2$, so
$\langle \sigma_e \sigma_f \rangle$ depends on the correlation through
a path of length~$g - 2$ around the cycle (the complement path).
Applying the transfer matrix method on~$C_g$ gives the exact
formula~\eqref{eq:adjacent-edge-cov}, in which the factor
$\tanh^{g-2}(J)$ reflects correlation decay along the complement path
and the denominator $(1 + \tanh^g(J))^2$ accounts for the cyclic
boundary condition.

For the operator norm bound, each off-diagonal entry
$F_{ef} = \operatorname{Cov}(\sigma_e, \sigma_f)$ decays as
$\tanh^{\ell}(J)$ where $\ell$ is the path length between the
corresponding vertex pairs through the shortest connecting route.
For edges sharing a vertex on a shortest cycle, $\ell = g - 2$;
for more distant edge pairs, $\ell \geq g - 2$.
By Gershgorin's theorem, the operator norm of the off-diagonal part
is bounded by the maximum absolute row sum:
$\|F - \operatorname{diag}(F)\|_{\text{op}} \leq
\max_e \sum_{f \neq e} |F_{ef}|$.
Each edge $e$ has at most $2(\Delta - 1)$ adjacent edges, and each
contributes at most $O(\tanh^{g-2}(J))$ to the row sum.
Since $\|\operatorname{diag}(F)\|_{\text{op}}$ is bounded below by the
minimum edge variance (which is order one in the Dobrushin regime),
dividing yields~\eqref{eq:near-diagonal-bound}
with~$C(\Delta) = 2(\Delta - 1)$.

\textit{Numerical verification.}
On cycle graphs $C_g$ ($g = 3, \ldots, 12$), the ratio
$C = \text{ratio}/\tanh^{g-2}(J)$ has coefficient of
variation $< 0.28$ across all tested~$J$, compared to $> 1.6$ for
the na\"ive exponent~$g$
(50~configurations; see~\texttt{near\_diagonal\_bound\_derivation.py}).
On general graphs (grids, bipartite, Petersen, Heawood), the bound
holds with larger constants reflecting the richer cycle structure.
\end{proof}

\begin{remark}[Universality across exponential families]
\label{rem:fisher-universality}
The near-diagonal property and spectral gap selection were tested
on $q$-state Potts models ($q = 2, 3, 4, 5$; 72~configurations),
Gaussian graphical models (51~configurations), and restricted
Boltzmann machines (RBMs; 27~configurations).
\emph{Potts models}: Tree Fisher Identity holds (40/40 trees, all~$q$),
$M = F^2$ holds (72/72 cases), and spectral gap selects $q_{\mathrm{neg}} = 1$
in $94.4\%$ of cases---all three are \emph{universal} across
discrete state Potts models.
\emph{Gaussian models}: Tree Fisher Identity \emph{fails}
(F is non-diagonal on trees), spectral gap selection shows no
preference (${\sim}55\%$, barely above chance), but $M = F^2$ and
the PSD obstruction still hold as general exponential family properties.
\emph{RBMs} (visible marginal of an exponential family, hence
\emph{not} itself exponential family): Tree Fisher fails ($11\%$
diagonal) and spectral gap selection fails ($22\%$).
The $M = F^2$ identity was not independently testable for RBMs
because the mass tensor $M = J^T J$ requires computing the Jacobian
of the mean-parameter map, which for marginal models differs from~$F$
(the sufficient statistics depend on parameters through the
conditional distribution).
Thus the near-diagonal structure and Lorentzian signature selection
are specific to \emph{discrete spin models}, not universal across
all exponential families; the physically relevant case of
discrete spacetime graphs falls in the favourable class.
\end{remark}

We verified these results numerically across 199~Ising Fisher matrices
spanning $N = 3$ to~$12$ (7~graph topologies, multiple coupling strengths
and random seeds):

\begin{table}[ht]
\centering
\caption{Spectral gap weighting $W(q=1)$ vs.\ $\max_{q \geq 2} W(q)$
across graph topologies ($N = 3$--$12$).
Sparse topologies show $100\%$ Lorentzian
preference; complete graphs fail at strong coupling.}
\label{tab:spectral-gap}
\begin{tabular}{@{}lccc@{}}
\toprule
Topology & $q = 1$ wins & Total cases & Percentage \\
\midrule
Path (tree, $N \leq 12$)    & 24 & 24 & 100\% \\
Star (tree, $N \leq 12$)    & 24 & 24 & 100\% \\
Cycle ($N \leq 12$)         & 24 & 24 & 100\% \\
Random tree-like             & 53 & 55 & 96\% \\
Random sparse (w/ cycles)    & 27 & 45 & 60\% \\
Complete ($K_3$--$K_{12}$)  & 16 & 27 & 59\% \\
\midrule
\textbf{Total} & \textbf{168} & \textbf{199} & \textbf{84\%} \\
\bottomrule
\end{tabular}
\end{table}

This should be compared to the success rate for \emph{generic} random
positive definite matrices: only $19\%$ (100~random PD matrices,
$m = 3$ to~$9$).
The $84\%$ vs.\ $19\%$ discrepancy is explained by the near-diagonal
structure of Ising Fisher matrices on sparse graphs
(\cref{thm:tree-fisher,thm:near-diagonal-fisher}).
Restricting to sparse topologies (degree~$\Delta \leq 3$,
girth~$g \geq 4$) yields $100\%$ Lorentzian preference across all
$N$ and~$J$ tested.

The complete graph failures ($K_4$ at $J \geq 0.5$, $K_5$ at
$J \geq 0.3$, $K_N$ at progressively weaker $J$ for $N \geq 6$)
occur when the graph is dense enough that the
off-diagonal corrections overcome the spectral gap advantage.
The predicted crossover coupling is
$J_{\mathrm{crit}} \sim 1/\sqrt{\Delta}$,
where $\Delta$ is the maximum vertex degree; this scaling is confirmed
by the data to a factor of~${\sim}2$.

\begin{figure}[htbp]
\centering
\includegraphics[width=0.8\textwidth]{Fig6_spectral_gap_selection.png}
\caption{Spectral gap weighting $W(q = 1)$ versus $\max_{q \geq 2} W(q)$
for Ising Fisher matrices across five topology classes
(131~cases from an independent enumeration; see \cref{tab:spectral-gap}
for the full 199-case analysis).
Points above the diagonal correspond to Lorentzian ($q = 1$) preference.
All sparse topologies (trees, cycles, tree-like, sparse with cycles) lie
strictly above the diagonal ($100\%$); only dense complete graphs
($K_5$--$K_8$ at strong coupling) fall below.}
\label{fig:spectral-gap-scatter}
\end{figure}

\begin{figure}[htbp]
\centering
\includegraphics[width=0.75\textwidth]{near_diagonal_fisher_decay.png}
\caption{Exponential decay of Fisher matrix off-diagonal strength with
girth.
Measured ratio
$\|F - \operatorname{diag}(F)\|_{\text{op}} /
\|\operatorname{diag}(F)\|_{\text{op}}$ versus girth~$g$ for
cycle graphs $C_n$ ($n = 4, \ldots, 20$) at $J = 0.5$.
Trees ($g \to \infty$) achieve exact diagonality.
The natural scaling exponent is $g - 2$
(\cref{eq:adjacent-edge-cov}), with effective constant
$C(\Delta) = 2(\Delta - 1) = 2$ on cycles ($\Delta = 2$)
(\cref{thm:near-diagonal-fisher}).}
\label{fig:near-diagonal-decay}
\end{figure}

\begin{remark}[Physical interpretation]
\label{rem:sparse-lorentzian}
If physical observers have sparse internal interaction graphs
(bounded degree, few short cycles)---as expected for systems with local
interactions---then the Fisher metric is near-diagonal and
$W(q = 1)$ dominates.
This provides a \emph{structural explanation} for Lorentzian spacetime:
sparse observers naturally perceive one time dimension, not because of
any dynamical mechanism, but because the information geometry of
sparse interactions singles out exactly one spectrally isolated
direction.
\end{remark}

\begin{figure}[htbp]
\centering
\includegraphics[width=0.8\textwidth]{Fig3_specific_heat_vs_beta_c.png}
\caption{Specific heat versus $\beta_c$ for observer topologies O1--O6.
The thermodynamic phase transition (specific heat peak, marked by circles)
occurs at $\beta_{\mathrm{eff}}$ values systematically different from the
geometric transition $\beta_c$ (diagonal line). This confirms that $\beta_c$
characterizes metric signature change, not thermodynamic criticality. Data
from Ising/Boltzmann exact enumeration on 2-parameter Gaussian model.}
\label{fig:specific-heat}
\end{figure}

\begin{figure}[htbp]
\centering
\includegraphics[width=0.8\textwidth]{Fig4_learning_dynamics.png}
\caption{Learning dynamics qualitative change at $\beta_c$. Natural gradient
trajectories with metric $g = M^{(\mathrm{H1'})} + \beta F$ show three
regimes: (bottom) unstable amplification along timelike direction for
$\beta < \beta_c$ (Lorentzian), (middle) singular behavior at
$\beta = \beta_c$ (degenerate metric), and (top) smooth convergent descent
for $\beta > \beta_c$ (Riemannian). Trajectories shown for observer O3
(path graph) initialized at $\theta_0 = (0.5, 0.5)$ on a 2-parameter
Gaussian model with simple quadratic loss.}
\label{fig:learning-dynamics}
\end{figure}

%----------------------------------------------------------------------
\section{Discussion}
\label{sec:discussion}
%----------------------------------------------------------------------

\subsection{What the failure means for discrete quantum gravity}
\label{sec:failure-implications}

The failure of the Lovelock bridge is not unique to the Wolfram--Vanchurin
connection.
The discrete-to-continuous symmetry barrier (\cref{sec:symmetry-barrier})
affects all discrete approaches to quantum gravity---loop quantum gravity,
causal set theory, causal dynamical triangulations---that must recover
smooth diffeomorphism invariance from discrete combinatorial structures.

Our numerical methodology (curvature convergence testing across rule
families, \cref{sec:numerical}) is transferable to other discrete
programs.
The $\kappa \sim 1/N$ scaling law for expanding rules
(\cref{tab:curvature}) may reflect a generic obstruction for discrete
approaches rather than a Wolfram-specific artifact, though we cannot
confirm this without testing other discrete frameworks.

\subsection{The genuine bridge: regime identification}
\label{sec:genuine-bridge}

Despite the failure of the Lovelock bridge, a genuine structural
connection exists between the two programs at a different level.

\begin{proposition}[Regime identification]
\label{prop:regime-identification}
In the $\alpha = 1$ (quantum) regime of Vanchurin's
framework~\cite{Vanchurin2025}, the covariant gradient descent equation
reduces to natural gradient descent.
Independently, for hypergraph observers satisfying the Good Regulator
conditions~\cite{companion_amari}, the learning dynamics are governed by
the natural gradient on the Fisher manifold.
The two derivations---one from Type~II metric theory, the other from
information-theoretic observer theory---converge on the same dynamical
equation.
\end{proposition}

This regime identification constitutes the real bridge between Programs~A
and~B: not at the level of spacetime geometry (which requires the failed
continuum limit), but at the level of observer dynamics (which operates
entirely within continuous parameter space).
We note that the multiway graph framework of
Gorard et al.~\cite{GorardNamuduriArsiwalla2021} provides a
categorical setting in which such regime identifications can be
formalized as functors between the multiway evolution categories of
the two programs.

\subsection{Implications of partial causal invariance for modified gravity}
\label{sec:partial-CI}

The negative results of \cref{sec:failure} concern the case where full
CI is assumed.  A natural question is: what gravitational theory
emerges when CI is only \emph{partially} satisfied?

The connection between discrete non-commutativity and torsion is
well-established in lattice gravity: in Regge calculus, torsion arises
from deficit angles not accounted for by curvature
alone~\cite{Caselle1989,Drummond1986}; in lattice gauge theory, torsion
corresponds to the failure of the parallelogram closure condition for
frame fields~\cite{MenottiPelissetto1987}.

We conjecture that an analogous mechanism operates in hypergraph
rewriting: when CI is violated, the non-commutativity of rewriting
orderings produces, in any putative continuum limit, a connection that
is not Levi-Civita.

\begin{conjecture}[CI violation and metric-affine gravity]
\label{conj:CI-MAG}
If a hypergraph rewriting system with approximate CI (causal graphs
agree up to $O(\epsilon)$) admits a continuum limit, the resulting
effective gravitational theory belongs to the class of metric-affine
gravity theories~\cite{HehlMcCreaMielkeNeeman1995}, with torsion and
non-metricity of order~$\epsilon$.
The specific type of CI violation determines the type of modified
gravity: order-dependent branching produces Riemann--Cartan torsion,
rule-subset confluence produces preferred-frame
effects (Ho\v{r}ava--Lifshitz type), and uniform approximate CI
produces general metric-affine corrections.
In the limit $\epsilon \to 0$, the theory reduces to GR.
\end{conjecture}

\Cref{conj:CI-MAG} is not testable with current tools: the continuum
limit required for its evaluation is numerically falsified for generic
expanding rules (\cref{sec:numerical}).
We emphasize that the deviations from GR predicted by CI violation
(torsion, non-metricity) belong to a \emph{different} class of
modified gravity from those predicted by deviations from natural
gradient learning in Vanchurin's
framework~\cite{Vanchurin2025,Vanchurin2025CGD}, which produce
metric-dependent corrections without torsion.

\subsection{Open problems}
\label{sec:open-problems}

We list selected open problems, organized by topic:

\begin{enumerate}
\item \textbf{Physical origin of edge signs} (MEDIUM, substantially addressed).
  The sign assignment $s_e \in \{+1, -1\}$ in H1$'$ is currently imposed
  rather than derived.
  A systematic study tested four strategies on 26~graph topologies
  (52~configurations): (A)~bipartite graph coloring, (B)~Fiedler vector
  spectral clustering, (C)~information flow (assign $s_e = -1$ to
  the edge with maximum $F_{ee}$), and (D)~oracle (brute-force optimal).
  All four strategies produce $q = 1$ (Lorentzian signature) in
  $100\%$ of tested configurations.
  Strategy~C achieves $99.7\%$ of oracle quality on average
  (minimum $82.2\%$), with a clear physical interpretation: the edge
  carrying maximum Fisher information becomes the timelike direction.
  However, a 240-configuration contrast experiment
  (\cref{rem:signed-edge-phase-diagram}) reveals that while H1$'$ with
  Fiedler signs creates a valid Lorentzian regime ($\beta_c > 0$),
  the spectral gap weighting satisfies $W(q{=}1) = W_{\max}$
  (degeneracy with higher-$q$ signatures) rather than strict $q = 1$
  dominance.
  Thus, sign assignment achieves Lorentzian signature ($q = 1$ is
  \emph{accessible}), but an additional selection principle beyond
  spectral gap weighting may be needed to explain why $q = 1$ is
  preferred over other valid signatures.
  Confidence that $q = 1$ is selected for physically relevant (sparse)
  observers: $60$--$75\%$.

\item \textbf{CI and exponential family statistics} (HIGH, partially
  resolved).
  The FSF factorization (\cref{thm:FSF}) and $M = F^2$
  (\cref{thm:M-F-squared}) hold specifically for canonical exponential
  families.
  Comprehensive analysis of six theoretical routes shows that causal
  invariance \emph{alone} does not force exponential family statistics
  (explicit counterexamples exist: mixture models, non-parametric
  observers).
  However, CI combined with the maximum entropy
  principle~\cite{Jaynes1957}
  and fixed observer topology does yield exponential family structure
  at 55--65\% confidence.
  A \emph{causal typicality} conjecture---that for large CI systems
  with small observers, the Kullback--Leibler divergence from MaxEnt
  decreases as $O(1/N)$---was tested in a redesigned v3 experiment
  with local observers ($4{,}500$ experiments: system sizes $L = 8$--$16$,
  observer windows $w = 2$--$4$, CI vs.\ non-CI controls).
  CI-like dynamics improve first-order convergence rates ($71\%$ vs.\ $27\%$
  for non-CI) and dominate second-order KL comparisons ($76\%$
  CI-better), but do \emph{not} consistently produce lower first-order
  KL than non-CI controls ($31\%$ CI-better).
  The strong form of causal typicality (CI $\Rightarrow$ MaxEnt)
  is therefore likely false ($\sim 25\%$ confidence);
  CI improves convergence but does not force it.
  A weaker version conditioning on global
  constraints remains open ($\sim 40\%$ confidence), analogous to
  canonical typicality in statistical
  mechanics~\cite{PopescuShortWinter2006}.
  Accepting MaxEnt as an independent epistemic axiom (55--65\%
  confidence) is currently the most viable route.
  This is the most important theoretical question for the program.

\item \textbf{Extension beyond two parameters} (LOW, rigorously proven for
  trees).
  The Tree Fisher Identity and absolute Lorentzian dominance are proven
  for arbitrary~$n$ (verified up to $n = 20$; all off-diagonal entries
  $< 10^{-15}$, diagonal values match $\operatorname{sech}^2(J)$ to
  machine precision).
  General-$n$ analysis (111~cases, $n = 3$ to~$20$, 5~topologies,
  $J \in \{0.1, 0.5, 1.0\}$) yields $98.2\%$ Lorentzian selection rate:
  trees $100\%$ (84/84), cycles $100\%$ (24/24),
  near-diagonal $100\%$ (81/81).
  The two failures occur at $n = 4$ with dense graphs ($K_4$ at
  strong coupling).
  The success rate is independent of~$n$ for sparse topologies.

\item \textbf{Dynamical sign selection} (MEDIUM, preliminary results).
  Gradient descent on the energy functional
  $E(\sigma) = (q(\sigma) - 1)^2$ with soft signs $\sigma_e \in [-1,1]$
  evolves toward $q = 1$ on $K_3$ (98--100\% success for $\beta < \beta_c$)
  but fails on $K_4$ ($\sim\!0$\% for Lorentzian energy).
  The mechanism does not scale to larger observer topologies in its
  current form.

\item \textbf{Spatial dimensionality from information geometry} (LOW,
  negative result).
  Testing whether information-geometric properties of the Fisher metric
  select a preferred spatial dimension~$d$, we computed spectral gap
  selection, near-diagonal ratios, and $\beta_c$ for Ising observers on
  $d$-dimensional lattice graphs ($d = 1, 2, 3, 4$; 16~configurations).
  Results: the near-diagonal ratio increases monotonically with~$d$
  ($0 \to 0.98 \to 1.82 \to 3.22$ at $J = 0.5$), and spectral gap
  $q = 1$ selection \emph{weakens} with dimensionality ($100\%$ at
  $d = 1$, $75\%$ at $d = 2$--$3$, $25\%$ at $d = 4$).
  There is no non-trivial distinction between $d = 3$ and other
  dimensions: the information geometry varies smoothly with~$d$
  without singling out $d = 3$ or $d = 3 + 1$.
  If the mechanism of \cref{sec:spectral-gap-selection} is correct,
  it selects Lorentzian signature (one time dimension) but is
  \emph{agnostic} about the number of spatial dimensions, which must
  be determined by other mechanisms (topology of the observer graph,
  embedding constraints, or dynamical considerations).

\item \textbf{Physical origin of $M = F^2$} (MEDIUM, partially resolved).
  The identity $M = F^2$ (\cref{thm:M-F-squared}) is proved
  algebraically for exponential families, but the question of
  \emph{which physical mechanism generates it} in actual observers
  was investigated by testing three hypotheses against the exact
  $M = F^2$ ground truth on 56~configurations (5~graph types,
  $N = 3$--$6$ nodes, $J \in \{0.1, 0.3, 0.5, 1.0\}$):
  (H1)~structural inertia ($M \propto F$ with best scaling,
  avg.\ error~$7.1\%$),
  (H2)~accumulated Fisher ($M$ as time-averaged Fisher along
  gradient flow, avg.\ error~$7.6\%$), and
  (H3)~graph connectivity ($M$ from graph Laplacian,
  avg.\ error~$73\%$, FAILED).
  H1 and H2 are near-exact on tree graphs (error~$< 0.3\%$) where
  $F = \operatorname{sech}^2(J)\,I$ makes $F^2 \propto F$,
  but degrade on cyclic and complete graphs (error~$15$--$37\%$)
  where off-diagonal structure breaks the proportionality.
  Graph Laplacian hypotheses (edge Laplacian squared, $B^T L^2 B$)
  fail universally.
  The physical mechanism that produces the exact $F^2$ structure on
  general graphs---rather than the approximate $\alpha F$ relationship
  valid only on trees---remains an open question for Vanchurin's
  Type~II framework.

\item \textbf{$M = F^2$ beyond exponential families} (LOW, open).
  Whether the identity $M = F^2$ extends to marginal models (restricted
  Boltzmann machines, mixture models) or latent variable models
  requires computing the Jacobian independently of~$F$, which is
  non-trivial for marginal distributions.
  Understanding the precise conditions under which $M = F^2$ holds
  would clarify the scope of the metric decomposition
  $g = M + \beta\,F$.
\end{enumerate}

\subsection{Honest novelty assessment}
\label{sec:novelty}

We assess the novelty of each contribution:

\begin{itemize}
\item \textbf{Negative results} (\cref{sec:failure}): 10--15\%.
  Documentation of known barriers with new numerical evidence
  (500-rule curvature analysis, up to $N = 18{,}508$) and systematic
  theoretical analysis.
  The barriers themselves (continuum limit, discrete symmetry) are
  known in the discrete quantum gravity community; our contribution
  is documenting their specific implications for the Wolfram--Vanchurin
  bridge.

\item \textbf{Mass-Fisher structure} (\cref{sec:mass-fisher}): $\sim 5\%$.
  Formalization of the metric decomposition and its consequences within
  Vanchurin's Type~II framework.
  The $M = F^2$ identity (\cref{thm:M-F-squared}) for canonical
  exponential families is a standard result in information geometry
  ($J = F$ for exponential families); our contribution is connecting
  it to Vanchurin's mass tensor.
  The spectral purity condition (\cref{thm:SRC}) is similarly a
  formalization within this framework rather than a standalone
  discovery.

\item \textbf{Lorentzian mechanism} (\cref{sec:lorentzian}): 15--20\%.
  The critical $\beta_c$ formula (\cref{thm:beta-c}), minimum complexity
  theorem (\cref{thm:min-complexity}), edge monotonicity
  (\cref{prop:edge-monotonicity}), non-equivalence result
  (\cref{thm:non-equivalence}), and persistence theorem
  (\cref{thm:persistence}) are, to our knowledge, new results with
  exact formulas and rigorous proofs.

\item \textbf{Spectral gap selection} (\cref{sec:spectral-gap-selection}):
  10--15\%.
  The Tree Fisher Identity (\cref{thm:tree-fisher}), the spectral gap
  bound (\cref{thm:Lgap-bound}), diagonal Lorentzian dominance
  (\cref{thm:diagonal-lorentzian}), perturbation stability
  (\cref{cor:lorentzian-perturbation}), the near-diagonal bound with
  exact adjacent-edge formula (\cref{thm:near-diagonal-fisher}), and
  universality across Potts models
  (\cref{rem:fisher-universality}) provide a conditional structural
  explanation for Lorentzian signature selection on sparse observer
  graphs (confidence 60--75\%; the signed-edge H1$'$ construction
  shows spectral gap degeneracy).
  The key insight---that information-geometric properties of the Fisher
  metric create Lorentzian \emph{accessibility} on sparse graphs without
  requiring a dynamical sign mechanism---appears new.

\item \textbf{Total estimated novelty}: 25--30\%.
  The nature of the contribution is: formal development of Type~II
  theory combined with honest negative results about the Type~I
  bridge and a conditional structural explanation for Lorentzian
  signature selection.
\end{itemize}

%----------------------------------------------------------------------
\section{Conclusion}
\label{sec:conclusion}
%----------------------------------------------------------------------

We set out to build a Lovelock bridge between Wolfram's hypergraph
physics and Vanchurin's neural network cosmology.
The bridge does not hold.

The continuum limit---the critical step from discrete to continuous
covariance---is empirically falsified for all dynamically nontrivial
hypergraph rewrite rules we tested.
The discrete-to-continuous symmetry barrier prevents passage from
Gorard's permutation-type invariances to the smooth Lorentz group.
And Vanchurin's evolution to a Type~II framework renders the bridge
unnecessary: the problem it was designed to solve (deriving continuous
covariance from discrete structure) has been bypassed by working in
continuous parameter space from the start.

The investigation was not fruitless.
By engaging seriously with Vanchurin's Type~II metric structure
$g = M + \beta\,F$, we obtained constructive results that contribute
to the formal development of this framework.

The critical $\beta_c$ formula (\cref{thm:beta-c}) provides an exact,
closed-form characterization of the Lorentzian--Riemannian transition
threshold.
The identity $M = F^2$ for exponential family models
(\cref{thm:M-F-squared}) establishes a definite algebraic relationship
between the mass tensor and information geometry.
The spectral purity condition (\cref{thm:SRC}) shows that perfect Good
Regulators inhabit conformally Fisher geometry.
The non-equivalence theorem (\cref{thm:non-equivalence}) clarifies the
relationship between our signed-edge construction and Vanchurin's
non-principal square root.
The edge monotonicity result (\cref{prop:edge-monotonicity}) and
persistence theorem (\cref{thm:persistence}) characterize how observer
topology and learning dynamics affect the Lorentzian regime.
The Tree Fisher Identity (\cref{thm:tree-fisher}), the Lorentzian
dominance theorem for diagonal Fisher matrices
(\cref{thm:diagonal-lorentzian}), and its perturbation stability
(\cref{cor:lorentzian-perturbation}) provide a conditional structural
explanation for why sparse observers may perceive exactly one time
dimension (confidence 60--75\%; see \cref{sec:open-problems}).
The near-diagonal bound with the exact adjacent-edge covariance
formula (\cref{thm:near-diagonal-fisher}) establishes that sparse
graphs fall in the near-diagonal regime where this dominance holds.
Universality across $q$-state Potts models ($q = 2$--$5$)
confirms that these results are not Ising-specific
(\cref{rem:fisher-universality}), while the failure for Gaussian
graphical models clarifies the scope: discrete spin models on sparse
graphs.
Whether the identity $M = F^2$ extends beyond exponential families
(e.g., to marginal models) remains open and requires independent
computation of the mean-parameter Jacobian.
The PSD obstruction (\cref{thm:psd-obstruction}) is validated
comprehensively: 240~configurations spanning observer sizes
$m = 2$--$15$, couplings $J = 0.1$--$2.0$, and three topologies
yield a universal $0\%$ Lorentzian win rate for $M = F^2$
(\cref{rem:psd-phase-diagram}), confirming the necessity of
signed-edge or non-principal square root constructions.
A systematic sign selection study (4~strategies, 52~configurations)
shows $100\%$ Lorentzian achievement with the information flow
strategy (maximum $F_{ee} \to$ timelike) at $99.7\%$ oracle quality.
However, a 240-configuration contrast experiment reveals that under
H1$'$ with Fiedler signs, the spectral gap weighting $W(q{=}1)$
equals $W_{\max}$ (degeneracy with higher-$q$ signatures), meaning
the Lorentzian regime is \emph{accessible} but not preferentially
selected by spectral gap alone
(\cref{rem:signed-edge-phase-diagram}).
The spectral gap mechanism is agnostic about spatial dimensionality:
tests on $d$-dimensional lattices ($d = 1$--$4$) show smooth
variation with~$d$ without singling out $d = 3 + 1$.

These results are offered as tools for further development of
learning-based spacetime emergence.
Multiple sign assignment strategies make Lorentzian signature
\emph{accessible}, and the PSD obstruction confirms that non-standard
mass tensors are necessary; the remaining open question is what
physical principle breaks the $q$-degeneracy to prefer $q = 1$.

Negative results in foundational physics are valuable.
Documenting \emph{why} the Lovelock bridge fails---with specific
numerical evidence and theoretical analysis---may save other
researchers from the same path and clarify the real obstacles in
connecting discrete and continuous approaches to spacetime emergence.

\section*{Acknowledgments}

The author thanks Vitaly Vanchurin for critical feedback on an earlier
version of this manuscript, which led to the reframing around
negative results and the Type~II constructive contributions.
Vanchurin's distinction between Type~I and Type~II
frameworks~\cite{Vanchurin2025} was essential in identifying why the
original Lovelock bridge fails and where productive work remains.

Independent research.
Computational resources: Apple M3~Max (128\,GB), Python (NumPy,
NetworkX, POT), Wolfram Language (SetReplace).
Code and data: \url{https://github.com/MaxZhuravlev/physics-cosmo-bridge}.

\bibliographystyle{unsrt}
\bibliography{references}

\end{document}
